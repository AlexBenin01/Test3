\section{Standard ISO/IEC 12207:1995}
Il gruppo ha deciso di applicare nelle proprie modalità di lavoro e quindi nel presente documento di adottare lo \textit{standard ISO/IEC 12207:1995 Information technology - Software life cycle processes}. In questa sezione del documento si possono trovare i criteri di applicazione e i processi definiti nell'ambito di questo standard.
\subsection{Processi del ciclo di vita del software}
Questo documento è usato per normare il \textit{way of working}\textsubscript{G} del gruppo, in particolare l'organizzazione dei processi del ciclo di vita del software\textsubscript{G} secondo lo \textit{standard ISO/IEC 12207:1995 Information technology - Software life cycle processes}, questi sono organizzati in una organizzazione gerarchica in cui ogni processo\textsubscript{G} è costituito da un insieme di attività, le quali possono prevedere delle procedure e un elenco di strumenti usati per lo svolgimento.
\subsubsection{Processi primari}
Lo standard adottato presenta cinque processi primari (\textit{Acquisizione, Fornitura, Sviluppo, Operazione, Manutenzione}), ma all’interno del contesto del progetto universitario in atto, il gruppo non ritiene i processi di \textit{Operazione} e \textit{Manutenzione} pertinenti, mentre il processo di \textit{Acquisizione} è di competenza del committente, pertanto, il gruppo decide di escluderli dalla presentazione nel documento.
I processi primari presentati nel presente documento sono:
\begin{itemize}
    \item \textbf{Fornitura}: definisce le attività del fornitore, l’organizzazione che fornisce il prodotto software all’acquirente;
    \item \textbf{Sviluppo}: definisce le attività dello sviluppatore, l’organizzazione che definisce e sviluppa il prodotto software.
\end{itemize}

\subsubsection{Processi di supporto}
I processi di supporto presentati nel presente documento sono:
\begin{itemize}
\item \textbf{Documentazione}\textsubscript{G}: definisce le attività per la registrazione delle informazioni prodotte da un processo del ciclo di vita;
\item \textbf{Configuration Management}\textsubscript{G}: definisce le attività di gestione della configurazione;
\item \textbf{Accertamento di qualità}: definisce le attività per assicurare in modo oggettivo che i prodotti e i processi software siano conformi ai requisiti\textsubscript{G} specificati;
\item \textbf{Verifica}\textsubscript{G}: definisce le attività per verificare il prodotto software;
\item \textbf{Validazione}\textsubscript{G}: definisce le attività per validare il prodotto software;
\item \textbf{Risoluzione dei problemi}: definisce un processo per analizzare e risolvere i problemi di qualsiasi natura o origine, sorti durante l’esecuzione di processi.
\end{itemize}


\subsubsection{Processi organizzativi}
I processi organizzativi presentati nel presente documento sono:
\begin{itemize}
\item \textbf{Gestione organizzativa}: definisce le attività dell’acquirente, l’organizzazione che acquisisce un prodotto software;
\item \textbf{Infrastruttura}: definisce le attività del fornitore, l’organizzazione che fornisce il prodotto software all’acquirente;
\item \textbf{Miglioramento}: definisce le attività dello sviluppatore, l’organizzazione;
\item \textbf{Formazione}: definisce le attività atte a provvedere una adeguata formazione del gruppo.
\end{itemize}

\subsubsection{Ruoli}
I ruoli definiti all’interno di questo progetto didattico universitario sono:
\begin{itemize}
\item \textbf{Docente del corso}: committente\textsubscript{G};
\item \textbf{Azienda proponente}: cliente e mentore;
\item \textbf{Gruppo di lavoro}: fornitore.
\end{itemize}