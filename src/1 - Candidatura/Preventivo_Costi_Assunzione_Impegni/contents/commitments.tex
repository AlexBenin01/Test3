


\section{Impegni individuali}
 Segue il riepilogo della distribuzione oraria rendicontata, che tiene conto dell'impegno orario stimato e di alcune preferenze espresse dai componenti del gruppo sulla base delle proprie conoscenze o altro, seguendo le seguenti regole:
    \begin{itemize}
        \item le ore lavorative dovranno essere le stesse per ogni componente (92);
        \item ogni componente dovrà ricoprire almeno una volta ogni ruolo;
        \item ogni componente dovrà assumenre un ruolo per un congruo numero di ore.
        
    \end{itemize}

\begin{table}[H]
    \centering
    \renewcommand{\arraystretch}{1.5}
    \arrayrulecolor{black} % Colore del bordo della tabella
    % Definiamo la colonna grigio chiaro per la prima e l'ultima colonna
    \begin{tabular}{|>{\bfseries}c|c|c|c|c|c|c|>{\bfseries}c|}
        \hline
        % Prima riga - Grigio molto scuro, testo bianco in grassetto
        \rowcolor{gray!70} 
        \color{white}\textbf{Nome} & \color{white}\textbf{Re} & \color{white}\textbf{Am} & \color{white}\textbf{An} & \color{white}\textbf{Pg} & \color{white}\textbf{Pr} & \color{white}\textbf{Ve} & \color{white}\textbf{Totale} \\
        \hline
        % Righe alternate con bianco e grigio chiaro
        \color{black}\textbf{Alessandro Benin} & 9 & 7 & 11 & 18 & 25 & 22 & 92 \\ 
        \hline
        \rowcolor{gray!10} % Grigio molto chiaro per riga alternata
        \color{black}\textbf{Ion Bourosu} & 7 & 8 & 9 & 18 & 26 & 24 & 92 \\ 
        \hline
        \color{black}\textbf{Matteo Gerardin} & 11 & 7 & 10 & 18 & 23 & 23 & 92 \\ 
        \hline
        \rowcolor{gray!10} % Grigio molto chiaro per riga alternata
        \color{black}\textbf{Derek Gusatto} & 9 & 8 & 10 & 18 & 26 & 21 & 92 \\ 
        \hline
         \color{black}\textbf{Davide Martinelli} & 8 & 8 & 10 & 18 & 24 & 24 & 92 \\ 
        \hline
        \rowcolor{gray!10} % Grigio molto chiaro per riga alternata
        \color{black}\textbf{Matteo Piron} & 10 & 8 & 11 & 17 & 23 & 23 & 92 \\ 
        \hline
        \color{black}\textbf{Tommaso Zocche} & 10 & 8 & 10 & 18 & 23 & 23 & 92 \\ 
        \hline
        % Ultima riga - Grigio molto scuro, testo bianco in grassetto
        \rowcolor{gray!70} 
        \color{white}\textbf{Totale} & \color{white}64 & \color{white}54 & \color{white}71 & \color{white}125 & \color{white}170 & \color{white}160 & \color{white}644 \\ 
        \hline
    \end{tabular}
    
\end{table}