\subsection{Capitolato C2 - Vimar GENIALE}
     \subsubsection{Informazioni generali}
        \begin{itemize}
            \item \textbf{Titolo}: Vimar GENIALE
            \item \textbf{Proponente}: Vimar S.p.A.
            \item \textbf{Committente}: Prof. Vardanega T., Prof. Cardin R.
        \end{itemize}
     \subsubsection{Obiettivo}
    L’obiettivo del progetto è quello di creare un sistema di supporto utile agli installatori, ovvero coloro che montano e mantengono un impianto elettrico e/o domotico. L’app deve essere capace di reperire velocemente informazioni dei prodotti e delle funzionalità richieste dagli installatori grazie al supporto di LLM e una base di dati open source. Un normale utente dell’app deve poter essere capace di fare una richiesta in linguaggio naturale ma anche tecnico ed ottenere una risposta precisa testuale oppure una parte testuale e un’immagine di eventuali schemi elettrici dei prodotti.

     \subsubsection{Dominio Applicativo}
E’ pertanto richiesto lo sviluppo di un sistema di conversazione libera (una chat) dove le risposte possano essere salvate. Il capitolato chiede di realizzare, nello specifico, 3 componenti:
\begin{itemize}
    \item App web responsive: L'applicazione web permette agli utenti di porre domande in linguaggio naturale tramite un’interfaccia di chat. Le risposte includono non solo testo, ma anche schemi elettrici, immagini, e collegamenti a risorse.
    \item App server:  Il server applicativo è responsabile dell’estrazione automatizzata delle informazioni dai cataloghi Vimar, indicizzando i dati in un database relazionale, da cui può rispondere con precisione alle richieste degli utenti.
    \item Infrastruttura cloud-ready: Il sistema è progettato per essere scalabile e utilizzabile tramite cloud
\end{itemize}    
\subsubsection{Tecnologie}
    Il progetto utilizza:

\begin{itemize}
    \item Docker e Docker Compose: Utilizzati per implementare il principio di Infrastructure as Code, consentendo la creazione e la gestione di ambienti applicativi replicabili con un solo comando.
    \item Git: Utilizzato per il versionamento del codice sorgente, con repository pubblici su piattaforme come GitHub, BitBucket, GitLab.
    \item Modelli LLM: Utilizzo di modelli open source come Llama 3.1, Mistral, Bert, o Phi per la generazione di testo.
    \item Framework Consigliati: Flask (Python) per il back-end e Angular o VueJS per il front-end.
    \item Database Relazionali: PostgreSQL o Database NoSQL.
    \item  Scrapy per il web scraping e OCRmyPDF per l'estrazione di testo da PDF.
    \item Utilizzo di strumenti come GitHub Runners per automatizzare test e analisi del codice.
    \item  Utilizzo di strumenti come GitHub Copilot o Amazon Q per supportare il processo di sviluppo.
\end{itemize}
    \subsubsection{Punti di forza}
    \begin{itemize}
    \item Il progetto si basa su tecnologie moderne come intelligenza artificiale e modelli di linguaggio di grandi dimensioni (LLM), offrendo un valore aggiunto significativo agli installatori e migliorando l'efficienza operativa.
    \item La decisione di sviluppare un prodotto open source permette la condivisione della conoscenza e l'adozione da parte della comunità, favorendo un ecosistema più ampio attorno ai prodotti Vimar.
\end{itemize}
    \subsubsection{Punti deboli}
    \begin{itemize}
    \item La necessità di accedere e raccogliere dati dal sito web di Vimar può presentare sfide, specialmente se la struttura del sito cambia o se ci sono limitazioni nel download dei PDF, portando a potenziali interruzioni nel flusso di lavoro.
    \item Sebbene il sistema di conversazione naturale sia un punto di forza, le limitazioni nella comprensione del linguaggio naturale da parte del modello AI potrebbero portare a risposte imprecise o insoddisfacenti, specialmente se le domande non sono formulate in modo chiaro. 
\end{itemize}
    \subsubsection{Conclusioni}
    Questo capitolato è risultato fin da subito interessante e per la maggioranza dei membri del gruppo ha subito stimolato grande curiosità, il progetto è stato posto fin dall'inizio ai vertici della classifica stilata in base all'interesse del gruppo.
