\subsection{Capitolato C5 - 3Dataviz}
    \subsubsection{Informazioni generali}
        \begin{itemize}
            \item \textbf{Titolo}: 3Dataviz
            \item \textbf{Proponente}: Sanmarco Informatica S.p.A.
            \item \textbf{Committente}: Prof. Vardanega T., Prof. Cardin R.
        \end{itemize}
    \subsubsection{Obiettivo}
        Realizzare un’interfaccia web per la visualizzazione in forma tridimensionale di dati tramite istogrammi 3D e i relativi dati d’origine.
        I metodi tradizionali di visualizzazione dei dati comportano diverse limitazioni nel supportare decisioni aziendali efficaci. Per sopperire a queste limitazioni è nata la necessità di rappresentare dati complessi in formati visivi interattivi e intuitivi, che facilitino l’analisi rapida e la comprensione.
    \subsubsection{Dominio Applicativo}
        Il progetto riguarda la creazione di un’interfaccia web concepita per rappresentare dataset ampi e articolati in un ambiente grafico 3D navigabile e intuitivo, che permetta di interpretare velocemente informazioni complesse e prendere decisioni basate su dati visivi chiari e facilmente accessibili.
        Le funzioni che dovrà possedere questo sistema sono:
        \begin{itemize}
            \item Rotazione, zoom e pan del grafico per una visualizzazione completa dei dati.
            \item Auto-posizionamento per centrare automaticamente la vista.
            \item Possibilità di opacizzare o nascondere le barre con valori superiori o inferiori rispetto a quella selezionata.
            \item Possibilità di visualizzare solo i valori più alti e più bassi per facilitare il confronto.
            \item Possibilità di visualizzare un piano parallelo alla base che rappresenta il valore medio dei dati.
            \item Possibilità di nascondere o opacizzare le barre sopra o sotto il piano che rappresenta la media.
            \item Possibilità di visualizzare piani paralleli che mostrano la media di specifici elementi lungo un singolo asse (opzionale).
            \item Visualizzazione dei valori numerici esatti delle barre al passaggio del mouse.
        \end{itemize}
        Inoltre l’applicazione deve permettere future espansioni, come nuove modalità di reperimento dati e ulteriori funzionalità di interazione.
    \subsubsection{Tecnologie}
        Il progetto utilizza:
        \begin{itemize}
            \item Three.js per la creazione e la gestione di grafici tridimensionali interattivi basati su WebGL, facilitando la creazione e la navigazione del modello 3D.
            \item D3.js per visualizzazioni dinamiche e interattive dei dati in HTML5, SVG e CSS, ideale per manipolare i dati e creare grafici customizzati.
            \item React o Angular per la costruzione dell’interfaccia utente in modo modulare e reattivo, migliorando la gestione dell’interazione con i dati e la navigazione.
            \item Connessione a database tramite SQL o integrazione con REST API pubbliche o inserimento manuale tramite UI web per popolare il grafico 3D con dati.
        \end{itemize}
    \subsubsection{Punti di forza}
        \begin{itemize}
            \item L’approccio 3D rende i dati complessi più intuitivi e comprensibili, migliorando il processo decisionale.
            \item Le funzioni come rotazione, zoom, filtraggio e visualizzazione dei valori più alti/bassi arricchiscono l’esperienza utente e consentono analisi mirate.
            \item L’architettura proposta è progettata per gestire grandi dataset e supportare future espansioni, facilitando l’integrazione di nuove fonti dati e funzionalità.
            \item L’uso di Three.js, D3.js e framework come React o Angular assicura prestazioni elevate e consente un’interfaccia utente dinamica e reattiva.
        \end{itemize}
    \subsubsection{Punti deboli}
        \begin{itemize}
            \item L’uso combinato di Three.js, D3.js e framework come React o Angular potrebbe risultare complesso, richiedendo tempo per apprendere e padroneggiare queste tecnologie.
            \item L’integrazione con API pubbliche e database esterni può introdurre problematiche di affidabilità e richiede attenzione per gestire eventuali interruzioni o modifiche nelle fonti dati.
            \item Visualizzare grandi volumi di dati in 3D può causare rallentamenti, richiedendo ottimizzazioni di performance significative.
        \end{itemize}
    \subsubsection{Conclusioni}
        Questo capitolato è stato considerato dalla quasi totalità dei membri tra le prime scelte del gruppo fin da subito, attirando l’interesse grazie al particolare tema trattato. Tuttavia è stato messo in ombra da altri capitolati che hanno riscosso un successo ancora maggiore, e hanno portato a considerarlo meno.


% à, è, ì, ò, ù,
% À, È, Ì, Ò, Ù

% á, é, í, ó, ú, ý
% Á, É, Í, Ó, Ú, Ý