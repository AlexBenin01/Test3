\section{Riassunto della riunione}
La riunione si è svolta secondo l’ordine del giorno previsto per le riunioni SAL:
\begin{itemize}
    \item \textbf{Presentazione dei ruoli:} 
    vengono elencati i ruoli assunti dai membri del gruppo
in questo periodo:
\begin{itemize}
        \item Alessandro Benin - Responsabile;
        \item Ion Bourosu - Analista;
        \item Matteo Gerardin - Amministratore;
        \item Derek Gusatto - Progettista;
        \item Davide Martinelli - Verificatore;
        \item Matteo Piron - Programmatore;
        \item Tommaso Zocche - Verificatore.
    \end{itemize}
\item \textbf{Panoramica ad ampio spettro}:
vengono riportati gli avanzamenti della documentazione, casi d'uso e analisi dei requisiti.
Vengono discusse le tecnologie frontend e backend, oltre a dei dubbi e difficoltà riscontrate nell'indicizzazione e nell'embedding.

 \item \textbf{Attività completate ed in corso}: 
 La documentazione è stata aggiornata e si continua a lavorarci, \`e stato iniziato il PoC. Il progettista e il programmatore hanno iniziato ad unire le tecnologie studiate e fare i primi test.
 Sono state riscontrate delle difficoltà in:
 \begin{itemize}
        \item Gestione di domande generiche rispetto a domande specifiche;
        \item Utilizzo di embedding per contestualizzare i dati;
        \item Problemi con Docker e LM Studio:
       \begin{itemize}
                    \item Errori relativi alla configurazione delle directory;
                    \item Divisione dei servizi in container separati (Postgres, Flask, LM Studio).
        \end{itemize}  
\end{itemize}
\item \textbf{Prossime attività da svolgere}:
\begin{itemize}
        \item PoC (Proof of Concept):
         \begin{itemize}
            \item Iniziato per dimostrare il funzionamento base;
            \item Completamento richiesto per l’RTB.
        \end{itemize}
        \item RTB (Relazione Tecnica di Base): Previsto tra il 15 gennaio e il 15 febbraio.
        \item Documentazione: Analisi dei requisiti e motivazioni delle scelte tecniche.
        \item Allineamenti futuri: Sessione in presenza per presentare i progressi architetturali.

    \end{itemize}


 \item \textbf{Discussione di dubbi e domande}:


\textbf{Come state gestendo le tecnologie come docker e l'integrazione con altri componenti?}\\
Abbiamo creato un Docker Compose con tre container separati: Postgres (con estensione vettoriale), Flask e LM Studio. Attualmente, stiamo lavorando per migliorare l'integrazione e studiare come esporre meglio le API. Alcune difficoltà riguardano la configurazione dell'ambiente e la documentazione incompleta per LM Studio.\\

\textbf{Qual è la strategia per trattare domande specifiche e generiche dell'utente?}\\
Il sistema implementa un pre-filtraggio delle domande dell'utente. In caso di domande generiche (es. ``parlami dei termostati"), il chatbot richiederà ulteriori dettagli. Per domande specifiche, verranno utilizzati embedding per identificare i dati pertinenti. In futuro, si pensa di implementare un sistema per confrontare prodotti o gestire domande relative a contesti precedenti.\\


\end{itemize}