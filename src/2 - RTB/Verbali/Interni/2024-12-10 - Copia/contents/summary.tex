\section{Riassunto della riunione}
\begin{itemize}
    \item Si decide di rimandare di un paio di giorni la rotazione dei ruoli per consentire il completamento di alcune attività e garantire un passaggio più efficiente. Il cambio di ruoli avverrà quindi alle 23.59 del 2024-12-12 con la seguente configurazione:
    \begin{itemize}
    \item \textbf{Responsabile}: Alessandro Benin 
    \item \textbf{Amministratore}: Matteo Gerardin 
    \item \textbf{Verificatori}: Tommaso Zocche, Davide Martinelli
    \item \textbf{Analista}: Ion Bourosu
    \item \textbf{Programmatore}: Matteo Piron
    \item \textbf{Progettista}: Derek Gusatto
    \end{itemize}
    \item Vengono consolidate alcune scelte a livello di tecnologie, nello specifico:
    \begin{enumerate}
        \item \textbf{Backend in Python}: come consigliato da Vimar S.p.A. il backend sarà programmato in Python, linguaggio semplice e veloce, molto aggiornato per i modelli di LLM, anche in ambito RAG ed embedding;
        \item \textbf{Database in PostgreSQL}: come suggerito da Vimar S.p.A. e dopo alcune prove si sceglie di usare PostgreSQL come DBSM grazie alla sua capacità e velocita in ambito di DB vettoriali (con l'estensione vector).
        \item \textbf{Modelli LLM}: in attesa di poter testare i modelli nel contesto adatto al fine di poterli valutare al meglio per il progetto, vengono esclusi Bert (limitato nel numero di token di contesto) e Phi (limitato alla lingua inglese) e rimangono in sospeso per la valutazione Llama3.1 e Mistral e openGPT-X;
        \item \textbf{Scrapy}: come consigliato da Vimar S.p.A. per il componente di estrazione e reperimento di informazioni dal sito web verrà utilizzato Scrapy in quanto molto efficiente.
    \end{enumerate}   
\end{itemize}