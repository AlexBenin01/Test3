\section{Riassunto della riunione}
Questa riunione si è svolta per allineare il gruppo in seguito alle attività di studio e prove pratiche indivituali per le tecnologie proposte nel capitolato. Da questa discussione sono emerse le seguenti decisioni:
\begin{itemize}
    \item Dopo la visione di alcuni script esemplificativi per le tecnologie proposte per la realizzazione del frontend, si è deciso di:
    \begin{enumerate}
        \item Escludere Vue.js, ritenuto il meno adatto alle nostre necessità essendo basato su JavaScript, linguaggio di programmazione nato per uno scopo che non coincide con le nostre necessità;
        \item Analizzare i punti a favore ed a sfavore di Angular e Flask, in maniera tale da poterli confrontare più precisamente e prendere un decisione definitiva sulla tecnologia da utilizzare per il frontend. 
    \end{enumerate}
    \item Rivedere lo script esemplificativo di integrazione tra un'applicazione web realizzata con Flask e lo script per l'interrogazione del modello Llama 3.1 realizzato in Python, migliorandolo o eventualmente sostituendo Flask con Angular;
    \item Dopo la visione di uno script esemplificativo per la realizzazione di un database vettoriale e di uno per il processo di web scraping, si è deciso di proseguire con l'approfondimento di queste tecnologie e di effettuare una prova di integrazione tra queste due componenti;
    \item Dopo un confronto tra i modelli Llama 3.1 e Mistral, si tende a preferir utilizzare il primo, in quanto è dotato di un maggior numero di token per il contesto utilizzato per fornire le risposte, richiesto per contenere le grandi quantità di informazioni che saranno inserite all'interno del database vettoriale, ma ci si riserva di fare ulteriori prove anche con il secondo, evitanto di scartarlo completamente per il momento;
    \item Effettuare un controllo delle librerie utilizzate per verificare la compatibilità con la licenza Apache 2, scelta dal gruppo.
\end{itemize}
Dopo aver trattato l'argomento principale della riunione, si è discusso anche riguardo il proseguimento della redazione dei vari documenti, prendendo le seguenti decisioni:
\begin{itemize}
    \item Analisi dei requisiti:
    \begin{enumerate}
        \item Revisione dei casi d'uso redatti fino a questo punto, riscrivendo quelli ritenuti poco chiari ed integrandoli tutti con i diagrammi UML;
        \item Integrazione degli attori con componenti ritenuti precedentemente parte del sistema e stesura dei relativi casi d'uso che riguardano questi nuovi attori;
    \end{enumerate}
    \item Inizio della redazione delle metriche di qualità nelle Norme di Progetto, di cui si sono evidenziate quelle principali e strettamente necessarie;
    \item Inizio della stesura del Piano di Qualifica;
    \item Continuazione della stesura della sezione periodi del Piano di Progetto.
\end{itemize}
Infine si è discussa la scelta di un nuovo giorno e di un nuovo orario per lo svolgimento delle riunioni SAL, richiesto dall'azienda a causa di conflitti con le riunioni SAL dell'altro gruppo che si è aggiudicato il capitolato Vimar GENIALE. Il gruppo ha deciso in maniera unanime che le prossime riunioni SAL verranno svolte il martedì alle ore 15:00 a partire dal giorno 10 dicembre 2024, mantenendo una durata degli sprint di 2 settimane.